\documentclass[11pt,a4paper,sans]{moderncv}

\moderncvtheme[blue]{classic}
\usepackage[utf8]{inputenc}
\usepackage[T1]{fontenc}
% last language in the list is the one used in the document
\usepackage[english,francais]{babel}
\usepackage{geometry}
\usepackage{moderntimeline}
\usepackage{multicol}
\usepackage{fontawesome}
\moderncvicons{awesome}
\usepackage{tikz}

% Configure page margins with geometry
\geometry{left=1.5cm, top=0.75cm, right=1.5cm, bottom=0.75cm, footskip=.5cm}

% Configure column width for timelines
\setlength{\hintscolumnwidth}{2cm}

% time interval for resume entries
\tlmaxdates{2010}{2017}
% width of the timeline 
\tlwidth{0.8ex}

% custom fonts for header 
\renewcommand*{\namefont}{\fontsize{45}{45}\mdseries\upshape}
\renewcommand*{\titlefont}{\fontsize{20}{25}\mdseries\upshape}
\renewcommand*{\addressfont}{\fontsize{10}{12}\mdseries\itshape}

% infos
\firstname{Gaël}
\familyname{Foppolo}  
\title{Étudiant ingénieur en informatique}         
\address{}{\faHome~Montpellier -- France}   
\email{me@gaelfoppolo.com}                      
\homepage{www.gaelfoppolo.com}
\mobile{+33 6 28 91 02 66}
\social[github]{gaelfoppolo}
\social[linkedin]{gaelfoppolo} 
\social[twitter]{gaelugio} 
\extrainfo{20 ans -- Permis B}
\photo[95pt][0pt]{ressources/avatar_nobg}	

\begin{document}

\maketitle

\vspace*{-7mm}

\section{\faBriefcase~Expérience}

\tldatecventry{2015}{Stage - Développeur iOS \faApple}{KeepCore}{Montpellier}{\url{www.keepcore.com}}{L'application \textit{HandiCarParking} a pour objectif de localiser rapidement les places de stationnement réservées aux personnes handicapées dans le monde entier. Elle se base sur les données d'\textit{OpenStreetMap}. L'application est disponible en plusieurs langues, gratuitement sur l'App Store.
\newline{\url{https://itunes.apple.com/fr/app/id986777305}}}

\section{\faGavel~Réalisations}

\tldatecventry{2014}{Pierre Feuille Ciseaux Lézard Spock \faHandRockO~\faHandPaperO~\faHandScissorsO~\faHandLizardO~\faHandSpockO}{Institut Universitaire de Technologie}{Montpellier}{\url{www.gaelfoppolo.com/projets/pfcls}}{Le projet s’appuie sur le jeu non coopératif \textit{Pierre Feuille Ciseaux Lézard Spock} en y associant l’âge et le sexe du joueur de façon à mettre en lumière des corrélations cachées dans les données ou des tendances de jeu générales. Nous proposons d’extraire des motifs de la forme : \textit{\og après avoir joué la figure A suivi de la figure B, les hommes ayant entre 18 et 22 ans ont tendance à jouer la figure C\fg}.
\newline{}}

\tldatecventry{2014}{SporkADE - Emploi du temps universitaire}{}{}{\url{www.sporkade.gaelfoppolo.com}}{Création d'un site convertissant son emploi du temps universitaire depuis ADE, l'application de gestion des emplois du temps des universités, vers un format ICS. Il permet l'exportation vers un logiciel de calendrier et une synchronisation sur smartphones.}

\section{\faGraduationCap~Formations}
\tlcventry{2015}{0}
{Ingénieur informatique}{Polytech}{Marseille}{}{}

\tlcventry{2013}{2015}
{DUT Informatique}{Institut Universitaire de Technologie}{Montpellier}{}{}

\vspace*{-2mm}
\section{\faNewspaperO~Publication}

\begin{otherlanguage}{english}

\tldatecventry{2015}{Contextual Sequential Pattern Mining in Games: Rock, Paper, Scissors, Lizard, Spock}{Research and Development in Intelligent Systems XXXII: Incorporating Applications and Innovations in Intelligent Systems XXIII}{Springer}
{Max Bramer}{}

\end{otherlanguage}

\vspace*{-2mm}
\section{\faLanguage~Langues}

\vspace*{-3mm}
\hspace*{\fill}
\begin{minipage}{\maincolumnwidth}
  \begin{multicols}{2}
    \centering{\textbf{Français\\}}
    \vspace*{2mm}
    \centering{\faStar~\faStar~\faStar~\faStar~\faStarHalfO\\}
    \centering{\textit{Projet Voltaire -- 80\%}}\\
    \columnbreak
    \centering{\textbf{Anglais\\}}
    \vspace*{2mm}
    \centering{\faStar~\faStar~\faStar~\faStar~\faStarO\\}
    \centering{\textit{TOEIC -- XXX}}\\
  \end{multicols}
\end{minipage}

\section{\faGears~Compétences}

\hspace*{\fill}
\begin{minipage}{\maincolumnwidth}
  \begin{multicols}{3}
    \centering{\textbf{Environnements\\}}
    \vspace*{2mm}
    \centering{UNIX~~\textsc{\faStar~\faStar~\faStar~\faStarO~\faStarO}\\}
        \centering{OS X/iOS~~\textsc{\faStar~\faStar~\faStar~\faStar~\faStarO}\\}
    \centering{Windows~~\textsc{\faStar~\faStar~\faStar~\faStarHalfO~\faStarO}\\}
    \columnbreak
    \centering{\textbf{Langages\\}}
    \vspace*{2mm}
    \centering{Swift~~\textsc{\faStar~\faStar~\faStar~\faStarHalfO~\faStarO}\\}
    \centering{Java/C~~\textsc{\faStar~\faStar~\faStar~\faStarO~\faStarO}\\}
    \centering{PHP/SQL~~\textsc{\faStar~\faStar~\faStar~\faStarO~\faStarO}\\}
    \columnbreak
    \centering{\textbf{Outils\\}}
    \vspace*{2mm}
    \centering{UML~~\textsc{\faStar~\faStar~\faStar~\faStarO~\faStarO}\\}
    \centering{Git~~\textsc{\faStar~\faStar~\faStar~\faStarHalfO~\faStarO}\\}
    \centering{\LaTeX~\textsc{\faStar~\faStar~\faStar~\faStarO~\faStarO}\\}
  \end{multicols}
\end{minipage}

\section{\faBeer~Intérêts}

\vspace*{1mm}

\tikzset{
    cercle/.pic={
      \node [draw, thick, circle, minimum width=10pt] {\tikzpictext};
    },
  }
%\hspace*{\fill}
\begin{minipage}{\maincolumnwidth}
  \begin{tikzpicture}
  	\pic [pic text={\huge \faCoffee}] {cercle};
    \node[draw=none] at (0,-1.1) {Café};
    \pic [pic text={\huge \faTelevision}] at (20mm,0) {cercle};
    \node[draw=none] at (2,-1.1) {Séries};
    \pic [pic text={\huge \faHeadphones}] at (40mm,0) {cercle};
    \node[draw=none] at (4,-1.1) {Musique};
    \pic [pic text={\huge \faLeaf}] at (60mm,0) {cercle};
    \node[draw=none] at (6,-1.1) {Jardinage};
    \pic [pic text={\huge \faApple}] at (80mm,0) {cercle};
    \node[draw=none] at (8,-1.1) {Apple};
    \pic [pic text={\huge \faGamepad}] at (100mm,0) {cercle};
    \node[draw=none] at (10,-1.1) {Jeux vidéo};
    \pic [pic text={\huge \faFirefox}] at (120mm,0) {cercle};
    \node[draw=none] at (12,-1.1) {Firefox};
    \pic [pic text={\huge \faSunO}] at (140mm,0) {cercle};
    \node[draw=none] at (14,-1.1) {Soleil};
    \pic [pic text={\huge \faRebel}] at (160mm,0) {cercle};
    \node[draw=none] at (16,-1.1) {Star Wars};
  \end{tikzpicture}
\end{minipage}

\end{document}